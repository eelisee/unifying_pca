\documentclass{article}
\usepackage[utf8]{inputenc}
\usepackage{amsmath, amssymb, amsthm}
\usepackage{mathtools}
\usepackage{bbm}
\usepackage{bbold}
\usepackage{graphicx}
\usepackage{tikz}
\usepackage{geometry}
\geometry{margin=3.5cm}

\theoremstyle{definition}
\newtheorem{defn}{Definition}[section]  % Definitionen
\newtheorem{remark}{Remark}[section]    % Bemerkungen
\newtheorem{thm}{Theorem}[section]      % Theorem
\newtheorem{lem}{Lemma}[section]        % Lemma
\newtheorem{ex}{Example}[section]       % Example
\newtheorem{prop}{Proposition}[section]       % Example

\title{Unifying PCA and Linear Models via Statistical Structure and Semigroup Theory}
\author{Elise Wolf \\ Based on discussions  with Prof. Schlather and ideas from Schlather \& Reinbott (2021)}
\date{\today}

% Structure
%1. Introduction: Questions aimed to answer

%2.	Theoretical groundwork:
%	•	Linear regression and its statistical linear model, variable selection in linear model
%	•	PCA in its original interpretation of Pearson (motivation of different best fit)
%	•	Motivation of PCA / Derivation of Method

%3. Statistical interpretation of both
%	•	Statistical interpretation of PCA (statistical model)
%	•	Sufficient statistics in linear model & difference in interpretation in PCA (why the estimators differ even with seemingly equal models)

%4. Algebraic Framework:
%	•	Matrix model formulation with PCA and Regression as an Operator
%	•	Building a semiring structure 
%	•	algebraic classification of operator subsets and their statistical interpretation
%	•	Concidence and Divergence conditions
%	•	Proving coincidence of linear regression and PCA variable selection under specific assumptions

%5. Examples of Coincidence and contradictions
%	•	Correlated predictors
%	•	orthogonal predictors
%	•	loss of signal by protection?
%	•	heavy-tailed / stable distributions where classical variance is undefined

%6. Contributions based on Semiring Perspectives
%	•	Non-commutativity as a diagnostic tool
%	•	Sufficiency gaps
%	•	Geometric contradiction maps


% Rest - Ideen:
%	•	Semiring factorization of inference goals
%	•	Robust semiring extensions
%	•	Universal approximation via operator semirings
%	•	Dynamic semiring evolution

%6. Comparison with existing selection models
%	•	Forward selection
%	•	backward selection
%	•	Lasso and Ridge

%7. PCA-Regression

\begin{document}

\maketitle

\begin{abstract}

We study the relationship between variable selection in Principal Component Analysis (PCA) and in linear regression. Although both methods operate on the same underlying statistical model in the case of independent random predictor variables, their selection principles differ fundamentally. PCA selects directions by the best low-dimensional linear approximation of the data, while linear regression selects variables with maximal explanatory power for a fixed response variable $Y$. We provide a unified perspective on both by examining their foundations in statistical modeling, through the lens of sufficiency. We construct an algebraic framework based on semiring structure that bridges both approaches and investigate its algebraic properties, building on the formalism introduced by Schlather and Reinbott (2021). This algebraic formulation clarifies the structural relationship between different variable selection principles, showing that their apparent discrepancies arise from distinct notions of sufficiency within the underlying statistical model.
\end{abstract}

\tableofcontents


% Kapitel 1
\section{Introduction: Questions Aimed to Answer}

Linear regression and PCA are two tools in multivariate analysis. Both reduce dimensionality, but they do so under conceptually different paradigms. Linear regression is a statistical approach utilizing s well-defined statistical model. PCA, however, relies on the same statistical model, but is traditionally implemented as a numerical procedure.

This paper addresses the following guiding questions:
\begin{itemize}
    \item Why do variable selection procedures in PCA and linear regression produce different outcomes, even though both are grounded in the same statistical model?
    \item Under which structural conditions do PCA-based and regression-based variable selections coincide?
    \item How can the differences be explained and formalized in an algebraic framework?
    \item What are the implications for practical selection methods such as forward selection, backward selection, and regularized regression?
\end{itemize}

Our contribution is threefold: 
(i) we present the theoretical groundwork of linear regression and PCA as statistical models; 
(ii) we interpret their sufficient statistics and explain the divergence of their estimators; 
(iii) we develop an algebraic operator framework that unifies linear regression and PCA and provides formal conditions for coincidence or divergence. 
Examples and comparisons with classical selection methods illustrate the theory. 


% Kapitel 2
\section{Theoretical Groundwork}
\label{sec:theoretical-groundwork}

%	•	Linear regression and its statistical linear model, variable selection in linear model
%	•	PCA in its original interpretation of Pearson (motivation of different best fit)
%	•	Motivation of PCA / Derivation of Method

\subsection{Linear Regression and its Variable Selection}

\begin{defn}{(Linear Model)}

Let $(\Omega, \mathcal{F}, \mathbb{P})$ be a probability space. The linear model is defined as
\[ Y = X\beta + \varepsilon, \quad \varepsilon \sim (0, \Sigma), \]
with $X \in \mathbb{R}^{n \times p}$ fixed, $Y \in \mathbb{R}^n$, $\beta \in \mathbb{R}^p$ and $\Sigma \in \mathbb{R}^{p\times p}$ the covariance matrix.

The corresponding statistical model is the family of probability distributions
\[
\mathcal{P} = \{ \mathbb{P}_\theta : \theta = (\beta, \Sigma) \in \Theta \subseteq \mathbb{R}^p \times \mathcal{S}^n \},
\]
where \( \mathcal{S}_{+}^n \) denotes the set of all symmetric positive semi-definite \( n \times n \) matrices.
\end{defn}

In the special case where the covariance matrix is parameterized as
\[
C = \sigma^2 C_0,
\]
with known \( C_0 \in \mathbb{R}^{n \times n} \), the parameter becomes \( \theta = (\beta, \sigma^2) \in \mathbb{R}^p \times (0, \infty) \), and the statistical model becomes
\[
\mathcal{P} = \{ \mathbb{P}_{\beta, \sigma^2} : \beta \in \mathbb{R}^p, \sigma^2 > 0 \}.
\]

\paragraph{Remark.}
Linear regression is a method used to estimate the unknown parameter \( \beta \) in the linear model. We assume the covariance matrix of the errors is \(\Sigma = \sigma^2 I_n\) (i.e., the errors are uncorrelated and have equal variance). Then the Ordinary Least Squares (OLS) estimator is obtained by minimizing the residual sum of squares:
\[
\hat{\beta}_{\text{OLS}} := \arg\min_{\beta \in \mathbb{R}^p} \| Y - X\beta \|_2^2.
\]

The unique OLS solution is
\[
\hat{\beta}_{\text{OLS}} = (X^\top X)^{-1} X^\top Y,
\]
assuming \(X\) has full rank. Under the Gauss-Markov assumptions (errors with zero mean and covariance matrix \(\Sigma\) positive definite), the OLS estimator is the Best Linear Unbiased Estimator (BLUE).  

More generally, assuming \(X\) has full rank and \(\Sigma\) is positive definite, the generalized least squares (GLS) estimator is
\[
\hat{\beta}_{\text{GLS}} = (X^\top \Sigma^{-1} X)^{-1} X^\top \Sigma^{-1} Y.
\]

%If \(X\) is rank-deficient and thus $X^TX$ is singular or \(\Sigma\) only positive semidefinite, then certain linear combinations of the errors have zero variance (i.e., vectors in the null space of $\Sigma$), which is caused by redundancy in the information contained in the sample. One may replace it by the pseudoinverse $\Sigma^+$, which yields a generalized estimator that is no longer unique in the parameter space, but still provides a well-defined projection of $Y$ onto the column space of $X$.


\subsection{Pearson's original formulation of PCA}

In his original 1901 paper, Karl Pearson introduced PCA as a problem of optimal approximation. Given high-dimensional observations $x_{i,\cdot} \in \mathbb{R}^p$, $i=1,\dots,n$, the task is to find a $k$-dimensional linear subspace onto which the data can be projected with minimal loss of information.

Formally, let $X \in \mathbb{R}^{n \times p}$ be the data matrix with centered and standardized columns. Pearson’s problem can be written as
\[
\min_{U \in \mathbb{R}^{p \times k}, \, U^\top U = I_k} 
\sum_{i=1}^n \| x_{i,\cdot} - U U^\top x_{i,\cdot} \|^2,
\]
i.e.\ find the orthogonal projection $UU^Tx_{i,\cdot}$ onto a $k$-dimensional subspace that minimizes the total squared deviation between the original points and their projections, quantifying the information loss as the squared reconstruction error.

\paragraph{Low-rank reconstruction of $X$.} 

The Eckart–Young theorem states that the solution is given by projecting $X$ onto the subspace spanned by the first $k$ eigenvectors of the sample correlation (or covariance) matrix
\[
R = \tfrac{1}{n} X^\top X = Q \Lambda Q^\top,
\]
where $\Lambda = \operatorname{diag}(\lambda_1, \dots, \lambda_p)$ with $\lambda_1 \ge \dots \ge \lambda_p \ge 0$ 
and $Q = [q_1,\dots,q_p]$ orthonormal.

The associated principal components are defined as
\[
Z = X Q_k \in \mathbb{R}^{n \times k},
\]
which are the coordinates of the projected data in the $k$-dimensional subspace spanned by the eigenvectors $q_1, \dots, q_k$. Reconstruction in the original space is then obtained as
\[
\widehat{X} = Z Q_k^\top = X Q_k Q_k^\top, \qquad Q_k = [q_1,\dots,q_k],
\]
and called the reconstructed data matrix, which is precisely the orthogonal projection of $X$ onto the span of the leading eigenvectors, and by the Eckart–Young theorem this projection is the unique best rank-$k$ approximation of $X$ in the Frobenius norm.

\paragraph{Variance of principal components.} 
By construction, the columns of $Z$ are mutually orthogonal. The empirical variance of the $j$-th component $Z_j = X q_j$, $j = 1, \dots, k$ is
\[
\mathrm{Var}(Z_j) = \frac{1}{n} Z_j^\top Z_j = \frac{1}{n} (X q_j)^\top (X q_j) = \frac{1}{n} ||X q_j||^2 = q_j^\top \underbrace{\frac{1}{n} X^\top X}_{R} q_j = q_j^\top R q_j,
\]
where the empirical mean of $Z_j$ is zero, since $X$ has centered columns. Since $q_j$ is an eigenvector of $R$ with eigenvalue $\lambda_j$, this gives
\[
\mathrm{Var}(Z_j) = q_j^\top R q_j = q_j^\top (\lambda_j q_j) = \lambda_j.
\]

Maximizing this variance under the normalization constraint leads to the Rayleigh quotient problem
\[
\max_{v \in \mathbb{R}^p, \, \|v\|=1} v^\top R v.
\]
The solution is the eigenvector $q_1$ of $R$ corresponding to the largest eigenvalue $\lambda_1$. 
The associated principal component is
\[
z^{(1)} = X q_1.
\]

Subsequent principal components are defined iteratively by maximizing $v^\top R v$ 
subject to $\|v\|=1$ and orthogonality to the previous directions $q_1, \dots, q_{k-1}$. 
This yields the ordered eigenvectors $q_1, \dots, q_p$ of $R$.

In summary, PCA yields both the principal components $Z$, capturing the maximal variance in successive orthogonal directions, 
and the reconstruction $\widehat{X}$, which minimizes the total squared reconstruction error among all rank-$k$ approximations of $X$. Thus, in Pearson’s view, PCA is an approximation method. It seeks the best low-rank linear representation of the data in the least squares sense. 

% Kapitel 3
\section{Statistical Interpretation of Linear Regression and PCA}

%3. Statistical interpretation of both
%	•	Statistical interpretation of PCA (statistical model)
%	•	Sufficient statistics in linear model & difference in interpretation in PCA (why the estimators differ even with seemingly equal models)

\subsection{Lack of Statistical Model in PCA}

PCA is regarded as a numerical approach since the deep statistical model is often not explicitly defined. However, there exists a statistical model. From this perspective, PCA is no longer just a procedure to solve an optimization problem, but rather a parametric model. This interpretation allows us to formulate PCA in analogy to the linear model
\[
Y = X \beta + \varepsilon,
\]
while still being distinct in its formulation. Concretely, let $X \in \mathbb{R}^{n \times p}$ be a centered data matrix. In PCA, we do not posit an external response Y, but instead seek to explain X by a low-rank linear transformation of itself. Thus, PCA can be viewed as a statistical model of self-approximation. The data are modeled as their own low-rank linear reconstruction plus noise. That is, we consider
\[
X \approx X \beta + \varepsilon,
\]
where $\beta \in \mathbb{R}^{p \times p}$ is a low-rank matrix constrained to be an orthogonal projection and $\varepsilon$ is the residual noise. The statistical model then takes the form
\[
\mathcal{P} = \{ \mathbb{P}_\theta : \theta = \beta, \; X \sim \mathcal{N}(X \beta, \Sigma) \},
\]
with noise covariance typically chosen as $\Sigma = \sigma^2 I_p$ or proportional to the sample correlation matrix $R$.

The corresponding estimator in the Frobenius norm of $\beta$ is
\[
\hat{\beta} = \arg\min_{\beta} \| X - X \beta \|_F^2,
\]
which, under the orthogonality constraint, yields
\[
\hat{\beta} = Q_k Q_k^\top,
\]
and hence the familiar PCA projection
\[
\widehat{X} = X \hat{\beta} = X Q_k Q_k^\top.
\]

\subsection{Sufficient Statistic of the Linear Model vs. PCA}

Comparing the linear model and PCA, we observe that both can be framed within the same multivariate Gaussian model for the data, 
\((X,Y) \sim \mathcal{N}(\mu, \Sigma)\), yet they pursue fundamentally different inferential goals. Linear Regression focuses on prediction of \(Y\) given \(X\), while PCA aims at approximating \(X\) itself. To understand those despite their shared statistical model, we examine the sufficient statistics in the classical linear model and their relation to the estimator.

\begin{defn}
Let $X : \Omega \rightarrow \Omega'$ be a random variable. A measurable function $T: \Omega' \rightarrow \Omega''$ is called a \emph{statistic}. A statistic $T(X)$ is \emph{sufficient} for a parameter $\theta$ if the conditional distribution of $X$ given $T(X)$ does not depend on $\theta$. Formally, for a parametric family $(\mathbb{P}_\theta)_{\theta \in \Theta}$:
\[
\mathbb{P}_\theta(X \in A \mid T(X) = t) = \mathbb{P}(X \in A \mid T(X) = t), \quad \forall A \in \mathcal{A}, \theta \in \Theta.
\]
\end{defn}

\begin{ex}
Consider the classical linear model with independent $X_i \in \mathbb{R}^n, i = 1, \dots, p$,
\[
Y = X \beta + \varepsilon, \quad \varepsilon \sim \mathcal{N}(0, \sigma^2 I_n),
\]
with known design matrix $X \in \mathbb{R}^{n \times p}$ and parameter $\beta \in \mathbb{R}^p$. The joint density of $Y$ is:
\[
f_Y(y; \beta) = \frac{1}{(2\pi \sigma^2)^{n/2}} \exp\left(-\frac{1}{2\sigma^2} \| y - X\beta \|^2 \right).
\]

This density depends on $\beta$ only through the quadratic form
\[
\| y - X\beta \|^2 = y^\top y - 2\beta^\top X^\top y + \beta^\top X^\top X \beta.
\]
Hence, the likelihood function $\mathcal{L}(\beta; y)$ depends on $y$ only through $X^\top y$ and $X^\top X$. Therefore, the statistic
\[
T(Y) := (X^\top X, X^\top Y)
\]
is sufficient for $\beta$.

The corresponding estimator obtained by maximizing the likelihood (MLE) is:
\[
\hat{\beta}_{\text{MLE}} = (X^\top X)^{-1} X^\top Y,
\]
which coincides with the OLS estimator $\hat{\beta}_{\text{OLS}}$ and depends only on the sufficient statistic $T(Y)$.
\end{ex}

This shows how, in regression, the sufficient statistic compresses the data $Y$ without losing information about $\beta$. The estimator $\hat{\beta}$ is a function of that statistic $T(Y)$. The linear model is not a model created for variable selection. Instead, the every variable $X_i \in \mathbb{R}^n$, $i = 1, \dots, p$ is part of the model itself.

\paragraph{Observation.}
By contrast, the choice of variables is an integral part of PCA. The difference arises from the choice of sufficient statistics and the associated inferential goal. Even though PCA and linear regression share the same statistical model and therefore the same joint distribution (e.g., $(X,Y) \sim \mathcal{N}(\mu, \Sigma)$), PCA is not directly linked to a likelihood-based estimation problem. Whereas regression defines a conditional distribution to be modeled or predicted, PCA seeks directions in $X$ that maximize marginal variance. Thus, the sufficiency question for PCA is of a different nature. It concerns retaining structure and interpretability in a transformed coordinate system, rather than preserving information about a parameter. Consider PCA under the Gaussian model
\[
X_1,\dots,X_n \sim \mathcal{N}(\mu, \Sigma), \quad X_i \in \mathbb{R}^p.
\]
By the factorization theorem, the empirical mean and covariance,
\[
\bar{X} = \tfrac{1}{n} \sum_{i=1}^n X_i,
\qquad
S = \tfrac{1}{n}\sum_{i=1}^n (X_i - \bar{X})(X_i - \bar{X})^\top,
\]
are sufficient for $(\mu,\Sigma)$. In PCA, the data are usually centered, so $\bar{X} = 0$, and the sufficient statistic reduces to the sample covariance (or correlation) matrix $R$. Thus,
\[
T(X) = R
\]
is the sufficient statistic underlying PCA. PCA does not aim to preserve inferential information about a specific parameter, but rather to retain the structural variance captured by $S$. In PCA, $X$ itself is the object of approximation. The direction of approximation is not fixed beforehand and can be reinterpreted, as long as the optimization goal of maximizing variance is retained. 

Thus, PCA allows not only the direction $X \to Y$ with $X$ explanatory and $Y$ response as in linear regression, but in principle any direction in the space of predictors: for each $X_j$, $j=1,\dots,p$, or more generally any linear combination of the $X_j$, depending on the chosen optimization criterion (variance, correlation, etc.). Hence, the interpretation of PCA is fundamentally multivalent. It is not fixed by a target variable, but by the optimization criterion and the projection space. It is important to note that in the linear model all predictors \(X_j\) are part of the model by design, and variable selection (subset selection) is an external procedure applied on top of the model, whereas in PCA the selection of directions (principal components) is intrinsic to the method. The PCA estimator depends on the selection order of the predictors $X_j$, $j = 1, \dots, p$, or in other words the ordering of components, i.e. the two main strategies are:

\begin{itemize}
    \item \textbf{Forward selection}: starting with a null model and adding the principal components with the largest variance in a decreasing order,
    \item \textbf{Exhaustive selection}: starting with the full model and deleting the principal component with the smallest effect on variance maximization in increasing order.
\end{itemize}

This implies a larger searchspace for possible solutions, 
\[ \text{Searchspace}_{\text{PCA}} \supseteq \text{Searchspace}_{\text{Linear Regression}} \]
and equal, if and only if $p=1$ for the dimensionality of the PCA.

\paragraph{Interpretation.}
The main difference is the directionality of the linear model in regression. It defines a mapping $X \mapsto Y$ and thereby expresses prediction. PCA, by contrast, does not impose any directional relation. In PCA all directions in $X_j$, $j = 1, \dots p$ are potential candidates. Based on this analysis, we propose the following alignment.:

\begin{quote}
\textit{To reconcile the statistical models of linear regression and PCA, one should require that the sufficient statistic preserves the directional role of the original variables $X$, i.e. their interpretability as coordinates in the data space.}
\end{quote}


% Kapitel 4
\section{Algebric Framework}

%4. Algebraic Framework:
%	•	Matrix model formulation with PCA and Regression as an Operator
%	•	Building a semiring structure 
%	•	algebraic classification of operator subsets and their statistical interpretation
%	•	Concidence and Divergence conditions
%	•	Proving coincidence of linear regression and PCA variable selection under specific assumptions

We now develop a generalized matrix-based framework that captures both linear regression and PCA as specific instances of a broader class of models. The key idea is to encode modeling and variable selection operations through structured linear maps, while ensuring that sufficient statistics and interpretability of variables are preserved.

\subsection{Composite Modeling via Structured Matrices}
\label{subsec:operator-formulation} 

We adopt the algebraic operator viewpoint introduced in Schlather and Reinbott~\cite{reinbott2021} and represent both linear regression and PCA as linear operators acting on an augmented data vector. This representation makes precise how both procedures can be embedded into a single algebraic framework.

\begin{defn}[Augmented data vector] Let $Z = ( \varepsilon, X_1, \dots, X_{k-1} )^T \in \mathbb{R}^k,$ so that $p=k-1$ is the number of predictors denote the augmented vector that collects the noise component \(\varepsilon\) and the predictor variables \(X_1,\dots,X_{k-1}\).
\end{defn}

\begin{defn}[Operator class \(P\)] We consider a class \(P\) of \(k\times k\) real matrices with the following block structure:
\begin{equation} \label{eq:block-A}
A \;=\;
\begin{bmatrix}
A_\sigma & A_\beta \\[4pt]
0_{(k-1)\times 1} & A_\mu \mathbb{1}_{(k-1)\times(k-1)}
\end{bmatrix},
\qquad
A_\sigma \in \mathbb{R}_{\geq 0},\;
A_\beta \in \mathbb{R}^{1\times (k-1)},\;
A_\mu \in \mathbb{R}_{\geq 0}.
\end{equation}
Here $A_\mu$ and $A_\sigma$ are scalars, and $I_{k-1}$ denotes the $(k-1) \times (k-1)$ identity matrix. This scalar assumption is essential for the selector-closure properties below. 

The action of \(A\) on \(Z\) is the vector \(AZ\). The first coordinate of \(AZ\) plays the role of the modeled response:
\begin{equation} \label{eq:response}
y \;=\; (AZ)_{1}
\;=\; \sum_{i=1}^{k-1} A_{\beta,i}\, X_i \;+\; A_\sigma\,\varepsilon.
\end{equation}
Matrices in \(P\) therefore encode (i) how the noise is scaled, (ii) how predictors enter linearly into the response, and (iii) how predictors may themselves be transformed (via \(A_\mu\)).
\end{defn}

\paragraph{Interpretation of linear regression and PCA.}
\begin{itemize}
  \item \textbf{Linear regression.} The classical linear model arises by restricting attention to operators \(A\in P\) for which \(A_\mu\) is (or acts like) the identity on the predictor block (or is left unconstrained but does not implement a low-rank reconstruction). The response then follows Eq.~\eqref{eq:response} with \(A_\beta\) giving the regression coefficients and \(A_\sigma\) controlling the noise scale.
  \item \textbf{PCA (as projection / reconstruction).} PCA corresponds to choosing operators \(A\in P\) with \(A_\sigma=0\) and where the predictor-block \(A_\mu\) is a \emph{projection} (or low-rank reconstruction) on the predictor space, e.g.
  \[
  A_\mu \;=\; Q_r Q_r^\top,
  \]
  where \(Q_r\in\mathbb{R}^{(k-1)\times r}\) collects \(r\) orthonormal eigenvectors (principal directions) of the predictor covariance matrix. Thus the operator A implements a reconstruction of the predictor vector from its orthogonal projection $Q_r Q_r^\top$ onto the $r$-dimensional subspace. There is no explicit external response variable in the PCA interpretation.
\end{itemize}

\subsection{Building a Semiring structure}
\label{subsec:semiring-structure}

The operator formulation in Section~\ref{subsec:operator-formulation} provides a common ground for representing regression and PCA as linear maps. To analyze these procedures within one algebraic framework, we now introduce a \emph{semiring structure} on the class of operators $P$.

\begin{defn}[Semigroup]
A set $(S, \circ)$ with an associative binary operation $\circ$ is called a \emph{semigroup}. A semigroup fulfills the following properties:

\begin{itemize}
    \item $(S, \circ)$ is closed under composition.
    \item Matrix multiplication is associative: $(A \circ B) \circ C = A \circ (B \circ C)$ for $A, B, C \in S$.
    \item Inverses do not generally exist within $S$, so $S$ is not a group, but a semigroup.
\end{itemize}
\end{defn}

\begin{defn}[Semiring]
A set $(R, +, \circ)$  consisting of a matrix addition $+$ and composition $\circ$ on $R$ is called a \emph{semiring} if:
\begin{enumerate}
    \item $(R, +)$ is a commutative monoid (associativity, identity, commutativity) with identity element $0$,
    \item $(R, \circ)$ is a monoid (associativity, identity) with identity element $1$,
    \item Distributivity of $\circ$ over $+$ holds,
    \item $0 \circ A = A \circ 0 = 0$ for all $A \in R$.
\end{enumerate}

A semiring generalizes a ring by not requiring the existence of additive inverses.
\end{defn}

\paragraph{Semiring of structured operators}

Let $P \subset \mathbb{R}^{k \times k}$ denote the class of block matrices defined in \eqref{eq:block-A}. We equip $P$ with
\[
A \oplus B := A + B \quad \text{(matrix addition)}, \qquad
A \otimes B := A B \quad \text{(matrix multiplication)}.
\]

\begin{prop}[Semiring structure of $P$]
The set $(P, \oplus, \otimes)$ forms a semiring. Specifically:
\begin{enumerate}
    \item \textbf{Additive closure:} For any $A,B \in P$ it holds $A \oplus B \in P$.
    \item \textbf{Multiplicative closure:} For any $A,B \in P$ it holds $A \otimes B \in P$.
    \item \textbf{Identities:} $0$ and $I_{k\times k}$ the identity matrix serve as additive and multiplicative identities.
    \item \textbf{Distributivity:} $A \otimes (B \oplus C) = A \otimes B \oplus A \otimes C$ and $(A \oplus B) \otimes C = A \otimes C \oplus B \otimes C$ for all $A,B,C \in P$.
\end{enumerate}
\end{prop}

\begin{proof}

\textbf{1. Additive closure and commutative monoid:}
Let $A, B \in P$ with block structures
\[
A = \begin{bmatrix} A_\sigma & A_\beta \\ 0 & A_\mu \end{bmatrix}, \quad
B = \begin{bmatrix} B_\sigma & B_\beta \\ 0 & B_\mu \end{bmatrix}.
\]
Then
\[
A \oplus B = \begin{bmatrix} A_\sigma + B_\sigma & A_\beta + B_\beta \\ 0 & A_\mu + B_\mu \end{bmatrix}.
\]
Clearly, $A_\sigma + B_\sigma \ge 0$, $A_\mu + B_\mu \ge 0$, and the lower-left block remains zero. Hence $A \oplus B \in P$, showing additive closure.  
Matrix addition is commutative and associative in general, and the zero matrix $0$ serves as additive identity. Therefore, $(P, \oplus)$ is a commutative monoid.

\textbf{2. Multiplicative closure and monoid:}
Computing the product of two block matrices in $P$ gives 
\[
A \otimes B = AB =
\begin{bmatrix} 
A_\sigma B_\sigma & A_\sigma B_\beta + A_\beta B_\mu \\ 
0 & A_\mu B_\mu
\end{bmatrix}.
\]
The lower-left block remains zero. The upper-left scalar $A_\sigma B_\sigma \ge 0$, and $A_\mu B_\mu \ge 0$ since both factors are nonnegative. Thus $A \otimes B \in P$, verifying multiplicative closure.  
Matrix multiplication is associative in general, and the identity matrix $I_{k \times k}$ is in $P$, serving as multiplicative identity. Therefore, $(P, \otimes)$ is a monoid.

\textbf{3. Distributivity:} For any $A,B,C \in P$:
\[
A \otimes (B \oplus C) = A(B+C) = AB + AC = A \otimes B \oplus A \otimes C,
\]
\[
(A \oplus B) \otimes C = (A+B)C = AC + BC = A \otimes C \oplus B \otimes C.
\]
Thus multiplication distributes over addition from both sides.
\end{proof}


\subsection{Variable Selection via Subsemirings}

To describe variable selection inside the operator class $P$, define diagonal selectors
\[
D = \operatorname{diag}(d_1,\dots,d_{k-1}),\qquad d_j\in\{0,1\}.
\]
Setting \(A_\mu \mathbb{1}_{(k-1)\times(k-1)}= D\) implements hard selection of predictors. Only those \(j\) with \(d_j=1\) are kept. Equivalently, restricting the coefficients \(A_{\beta,j}\) to be zero for certain indices removes predictors from the linear relation of Eq.~\eqref{eq:response}.

\begin{defn}[$S_\ell$, $L_\ell$]
Following the construction in Schlather and Reinbott~\cite{reinbott2021}[Ex.\ 2.12], we introduce two families of operator-subsets that encode common selection patterns:
\[
\begin{aligned}
S_\ell &\;:=\; \{\, A\in P : A_{\beta,j}=0 \text{ for all } j\neq \ell \,\}, \\[3pt]
L_\ell &\;:=\; \operatorname{span}\{ S_1,\dots,S_\ell\}
      \;=\; \{\, A\in P : A_{\beta,j}=0 \text{ for all } j>\ell \,\}.
\end{aligned}
\]
Intuitively, \(S_\ell\) corresponds to a \emph{simple regression model} that uses only one predictor (the $\ell$-th), while \(L_\ell\) corresponds to the a \emph{restricted model} that allows any linear combination but only from a fixed subset of the first \(\ell\) indices.
\end{defn}

\begin{prop}[Subsemiring structure of $S_\ell$ and $L_\ell$]
Let $1 \le \ell \le k-1$. Then $S_\ell$ and $L_\ell$ are subsemirings of $P$ under
\[
A \oplus B := A + B, \qquad A \otimes B := AB.
\]
\end{prop}

\begin{proof}
We prove the claim for $L_\ell$; the argument for $S_\ell$ is analogous.

\textbf{1. Additive closure:}  
Let $A, B \in L_\ell$. By definition, $A_{\beta,j} = B_{\beta,j} = 0$ for all $j > \ell$. Then
\[
(A \oplus B)_{\beta,j} = A_{\beta,j} + B_{\beta,j} = 0 \quad \forall j > \ell,
\]
so $A \oplus B \in L_\ell$.

\textbf{2. Multiplicative closure:}  
Consider $A, B \in L_\ell$ with block form
\[
A = \begin{bmatrix} A_\sigma & A_\beta \\ 0 & A_\mu \end{bmatrix}, \quad
B = \begin{bmatrix} B_\sigma & B_\beta \\ 0 & B_\mu \end{bmatrix}.
\]
Then the $\beta$-block of the product is
\[
(A \otimes B)_\beta = A_\beta B_\mu + A_\sigma B_\beta.
\]
$A_\beta$ has zeros in positions $j>\ell$, and $B_\mu$ maps the predictor block to itself (upper-left $(k-1)\times(k-1)$), so $A_\beta B_\mu$ has zeros for $j>\ell$. $B_\beta$ has zeros for $j>\ell$ and $A_\sigma \ge 0$, so $A_\sigma B_\beta$ also has zeros for $j>\ell$. Hence $(A \otimes B)_\beta$ has zeros in positions $j>\ell$, i.e., $A \otimes B \in L_\ell$.

\textbf{3. Identities:}  
The zero matrix $0 \in L_\ell$ and the identity matrix $I_{k\times k} \in L_\ell$ (acting on positions $j>\ell$) serve as additive and multiplicative identities.
\end{proof}

This shows that variable selection can be encoded algebraically as restricting the operator $A$ to particular subsemirings of $P$, with $S_\ell$ representing \emph{single-predictor models} and $L_\ell$ representing \emph{restricted multi-predictor models}.

\subsection{PCA as an Operator-Choice Problem}
\label{subsec:pca-operator}

In Chapter~\ref{sec:theoretical-groundwork}, we have seen that PCA can be interpreted as a best low-rank approximation problem in the Frobenius norm, i.e.\ in the least-squares sense. We now reformulate this problem in the \emph{operator class} $P$ introduced in Section~\ref{subsec:operator-formulation}, allowing a unified algebraic treatment of linear regression and PCA.

Let \(P \subset \mathbb{R}^{k \times k}\) denote the class of structured block matrices representing linear operators as in \eqref{eq:block-A}. Classical PCA can then be expressed as a search over \emph{reconstruction operators} \(H \in P\) that minimize a reconstruction loss measured by a (semi-)metric \(\rho\):
\begin{equation} \label{eq:general-pca}
\operatorname{PCA}_r(X) \;=\; \arg\min_{\substack{H\in\mathcal{H}\\ \operatorname{rank}(H)\le r}} \rho(X,\, H X),
\end{equation}
where
\begin{itemize}
    \item \(\mathcal{H} \subseteq P\) is an admissible class of operators. In the Euclidean setting, this contains the orthogonal projectors \(H = Q_r Q_r^\top\), with \(Q_r \in \mathbb{R}^{(k-1) \times r}\) collecting the first \(r\) principal directions. The set $\mathcal{H}_r$ containing the orthogonal projections onto a subspace of rank $r$ is not a subsemiring under standard matrix multiplication (idempotent projections do not compose additively), but it is a semigroup under multiplication. Hence, the PCA operators naturally form a semigroup embedded in the same overarching structure \(P\).
    (\item \(\mathcal{H} \subseteq P\) is an admissible class of operators. In the Euclidean setting, this contains operators with rank-$\le r$ reconstruction blocks. The set of orthogonal projection operators $\mathcal{P}_r$ (rank-$r$ idempotent symmetric matrices) contains the PCA minimizers but is in general neither closed under addition nor under multiplication (unless special commuting conditions hold). By contrast, the larger class of operators with $\operatorname{rank}\le r$ is closed under multiplication (since $\operatorname{rank}(AB)\le\min\{\operatorname{rank}A,\operatorname{rank}B\}$) and therefore forms a semigroup. For algebraic analysis of operator composition it is useful to consider the rank-$\le r$ class as the semigroup that contains orthogonal projectors as extremal elements.)
    \item \(\rho : \mathbb{R}^{n \times (k-1)} \times \mathbb{R}^{n \times (k-1)} \to \mathbb{R}_{\ge 0}\) is a reconstruction loss, typically the Frobenius norm:
    \[
        \rho(X, HX) = \| X - HX \|_F^2 = \sum_{i=1}^n \| x_{i,\cdot} - H x_{i,\cdot} \|_2^2.
    \]
    This reproduces the standard PCA criterion of minimizing the total squared reconstruction error.
\end{itemize}

The operator \(H\) acts only on the predictor block \(X\) (i.e., \(A_\mu\) in the block notation), with \(A_\sigma = 0\) since PCA does not involve an explicit noise term in the reconstruction. In this sense, PCA is a \emph{constrained operator selection problem}: one searches for \(H \in P\) that is idempotent, low-rank, and provides the optimal reconstruction in \(\rho\). The classical solution \(H = Q_r Q_r^\top\) arises as the unique minimizer of \(\rho\) in the least-squares sense (Eckart–Young theorem).

\paragraph{Interpretation.}
From a statistical perspective:
\begin{itemize}
  \item The family \(\{L_\ell\}\) formalizes \emph{forward selection}, since enlarging \(\ell\) increases the number of predictors admitted into the model.  
  \item The family \(\{S_\ell\}\) can be interpreted as \emph{exhaustive selection}, if all permutations of predictors are included.
\end{itemize}
Thus, familiar selection strategies arise naturally as algebraic restrictions in the semiring.

\paragraph{Algebraic classification of operator subsets and their statistical interpretation}

The semiring $P$ contains a variety of algebraically distinguished subsets that correspond to different statistical procedures. They can be split into:

\begin{itemize}
  \item \textbf{Regression-type subsets.}  
  The families $S_\ell$ and $L_\ell$ are subsemirings of $P$. They encode regression models with restricted sets of predictors: $S_\ell$ corresponds to simple regression using a single predictor, while $L_\ell$ corresponds to all regression models involving predictors up to index $\ell$. Algebraically, these subsets are nested,
  \[
  S_1 \subset L_1 \subset L_2 \subset \dots \subset L_{k-1} \subset P,
  \]
  mirroring the inclusion of models with increasing numbers of predictors.

  \item \textbf{PCA-type subsets.}  
  The families $\mathcal{H}_r$, consisting of operators with a projection block $A_\mu$ of rank at most $r$, form semigroups but not subsemirings. Their nesting
  \[
  \mathcal{H}_1 \subset \mathcal{H}_2 \subset \dots \subset \mathcal{H}_{k-1}
  \]
  corresponds to projections onto subspaces of increasing dimension, i.e.\ principal components of higher order.

  \item \textbf{Hybrid subsets.}  
  Operators with both nontrivial regression part ($A_\beta\neq 0$) and projection block ($A_\mu$ low-rank) fall into the hybrid category. These include principal component regression and related methods, which combine supervised and unsupervised features.
\end{itemize}

Linear regression classes are closed under addition and multiplication (semiring), PCA classes are stable under composition (semigroup), and hybrid classes are at the intersection. The overlap or divergence of regression-type subsets and PCA-type subsets can now be studied using semiring-theoretic tools, e.g.\ whether certain \(L_\ell\) and \(\mathcal{H}_r\) coincide, intersect nontrivially, or generate the same semimodule of operators.

% Kapitel 5
\section{Examples of Coincidence and Contradictions}

%5. Examples of Coincidence and contradictions
%	•	Correlated predictors
%	•	orthogonal predictors
%	•	loss of signal by protection?
%	•	heavy-tailed / stable distributions where classical variance is undefined

Having identified regression-type and PCA-type operator families, we now analyze their intersections and divergences at the model level. Algebraically, coincidence occurs when a regression-type operator $A \in L_\ell$ can also be represented as a PCA-type operator $H \in \mathcal{H}_r$ (i.e., a low-rank reconstruction operator in the model algebra). Divergence occurs whenever the two operator types project the model algebra into distinct subspaces that preserve different structural relations among the predictors.

\paragraph{General principle.}  
\begin{itemize}
    \item Coincidence requires that a regression operator, which selects a subset of predictors via $A_\beta$ and $A_\mu$, coincides with a low-rank PCA operator $H$ that preserves the same model-level subspace of the predictor algebra.
    \item Divergence occurs whenever the PCA operator $H \in \mathcal{H}_r$ projects the model algebra into a subspace that does not align with the functional relations encoded by the regression operator $A_\beta$.
\end{itemize}

\paragraph{Interpretation.}
From the operator perspective, coincidence means that both procedures induce the same action on the model algebra. Regression selects the functional dependencies corresponding to the chosen predictors, and PCA projects the structural block $A_\mu$ onto a subspace that exactly reproduces these dependencies. Divergence reflects situations where PCA’s projection alters the structural relationships, for example by combining multiple predictors into a principal subspace that does not coincide with the regression-selected components.

\begin{prop}[Coincidence under diagonal covariance operator]
Let $\mu_\Sigma \in \mathcal{M}$ be a Gaussian model with covariance operator $\Sigma \in \mathcal{S}_{+}^{k-1}$, and assume $\Sigma$ is diagonal, i.e., all predictors are uncorrelated. Then:
\begin{enumerate}
    \item Each standard basis vector $e_j \in \mathbb{R}^{k-1}$ is an eigenvector of $\Sigma$ with eigenvalue $\lambda_j = \operatorname{Var}(X_j)$.
    \item The rank-$1$ PCA operator $H_1 = e_{j^*} e_{j^*}^\top \in \mathcal{H}_1$ that minimizes the model-level loss $\mathcal{L}(\mu_\Sigma, H_1 \mu_\Sigma)$ coincides with the regression operator $A \in L_1$ selecting the predictor $X_{j^*}$ with maximal variance $\lambda_{j^*}$.
    \item More generally, for rank $r$, the PCA operator
    \[
        H_r = \sum_{s=1}^r e_{j_s} e_{j_s}^\top \in \mathcal{H}_r
    \]
    coincides with a regression operator $A \in L_\ell$, where $\ell = r$ and the selected predictors correspond to the $r$ largest variances along the diagonal of $\Sigma$.
\end{enumerate}
\end{prop}

\begin{proof}
Diagonal covariance implies that the eigenvectors of $\Sigma$ coincide with the standard basis.  
The PCA operator $H_r$ projects the structural block $A_\mu$ onto the subspace spanned by the $r$ largest eigenvalues. By construction, this subspace aligns with the predictors selected by $L_\ell$. Therefore, the action of $H_r$ on the model algebra reproduces exactly the dependencies encoded by $A_\beta$. Hence, coincidence holds.
\end{proof}

This proposition formalizes the intuitive idea that regression-type operators and PCA-type operators coincide precisely when the structural block of the model operator is diagonal, so that the PCA projection preserves exactly the same predictor components selected by regression. In the presence of correlations among predictors, the PCA operator projects onto subspaces that combine multiple coordinates, causing the regression and PCA operators to act on different subspaces of the model algebra and thus generally diverge.

We now turn to concrete examples in which the two approaches either coincide or diverge. This serves two purposes: (i) it illustrates how the abstract operator classes $L_\ell$ (regression) and $\mathcal{H}_r$ (PCA) behave in practice, and (ii) it highlights the role of covariance structure, signal alignment, and distributional assumptions in determining when the two procedures can agree.

\paragraph{Orthogonal Predictors as a Trivial Semiring Case}

Suppose $X_1, \dots, X_p \sim \mathcal{N}(0,1)$ independent, and let $Y = X_1 + \varepsilon$, $\varepsilon \sim \mathcal{N}(0,1)$. Here the covariance matrix is diagonal, so PCA operators $H \in \mathcal{H}_r$ selects the coordinate axes. Linear regression selects $X_1$ as the relevant predictor. Thus, PCA and regression coincide, and both correspond to a semiring element $A \in L_k \cap \mathcal{H}_r$ that projects onto $X_1$. This example illustrates the trivial intersection of the linear regression semiring and the PCA semiring (since only one projector semigroup works as a semiring), both are aligned with the coordinate basis.

\paragraph{Correlated Predictors as a Contradiction}

Now let $X_2 = X_1 + \eta$ with $\eta \sim \mathcal{N}(0,0.01)$, and again $Y = X_1 + \varepsilon$. Then regression identifies $X_1$ as the unique source of signal, but PCA selects approximately $X_1 + X_2$, the direction of maximal variance. Algebraically, the regression operator $A_\beta$ is not contained in the image of $H \in \mathcal{H}_1$, so the PCA and regression operators diverge at the model level. The semiring element representing regression and the semigroup element representing PCA no longer coincide. This example demonstrates that the two structures, though unified algebraically, may generate incompatible models.

\paragraph{Loss of Signal under Projection}

Let $X = (X_1,X_2)$ with $\mathrm{Var}(X_1)\gg \mathrm{Var}(X_2)$ and $\beta=(0,1)$. Then PCA selects $X_1$ as first component, but the signal lies in $X_2$. In operator terms, $A_\beta$ is orthogonal to $\mathrm{Im}(A_X)$. Thus the regression functional is annihilated by the PCA projection. This illustrates how projection may erase inferentially relevant information, that cannot be captured by variance maximization of PCA alone.

\paragraph{Heavy-tailed Distributions}

If $X \sim S(\alpha)$ is $\alpha$-stable with $\alpha < 2$, the covariance matrix does not exist, and PCA cannot be defined in its classical form. Regression, however, can still be expressed through conditional expectation. This shows that the PCA semiring $\mathcal{H}_r$ is undefined, while the regression semiring $L_\ell$ remains meaningful. Hence, the semiring framework distinguishes between procedures that depend on second-order structure (PCA) and those that do not (regression). This provides a structural explanation for why PCA is not universally applicable.

%%%%%%%% OLD
%\section{Examples of Coincidence and Contradictions}

%Having identified regression- and PCA-type operator families, we now analyze their intersections and divergences. Algebraically, coincidence occurs when a regression-type operator $A \in L_\ell$ can also be represented as a PCA-type operator $H \in \mathcal{H}_r$ (i.e., a low-rank reconstruction operator). Divergence occurs whenever the two operator types select different subspaces of the predictor space.

%\paragraph{General principle.}  
%\begin{itemize}
%  \item Coincidence requires that a regression operator, which retains only a subset of predictors via $A_\beta$ and $A_\mu$, can be expressed as a low-rank reconstruction $H$ that preserves exactly the same subspace of predictors.
%  \item Divergence occurs whenever the PCA projection $A_\mu$ spans a subspace that is not aligned with the coordinate axes associated with the regression selection $L_\ell$.
%\end{itemize}

%\paragraph{Interpretation.}  
%From a data representation perspective, coincidence means that both procedures reconstruct exactly the same components of $X$. PCA chooses a principal direction that coincides with a coordinate axis, and regression restricts itself to the same predictor. Divergence reflects the difference between axis-aligned selection (regression) and subspace reconstruction (PCA), which may involve linear combinations of multiple predictors.

%\begin{prop}[Coincidence under diagonal covariance]
%Let $X=(X_1,\dots,X_{k-1})$ be centered predictors with covariance matrix $\Sigma \in \mathbb{R}^{(k-1)\times(k-1)}$.  
%Assume that $\Sigma$ is diagonal, i.e., $\operatorname{Cov}(X_i,X_j) = 0$ for $i \neq j$. Then:
%\begin{enumerate}
%    \item Each standard basis vector $e_j \in \mathbb{R}^{k-1}$ is an eigenvector of $\Sigma$ with eigenvalue $\lambda_j = \operatorname{Var}(X_j)$.
%    \item The rank-$1$ PCA operator $H_1 = e_{j^*} e_{j^*}^\top$ that minimizes the reconstruction loss over $\mathcal{H}_1$ coincides with the regression operator $A \in L_1$ selecting the predictor $X_{j^*}$ of maximal variance $\lambda_{j^*}$.
%    \item More generally, for any $r$, the PCA operator $H_r = \sum_{s=1}^r e_{j_s} e_{j_s}^\top \in \mathcal{H}_r$ coincides with $A \in L_\ell$, where $\ell = r$ and the selected predictors correspond to the $r$ largest variances.
%\end{enumerate}
%\end{prop}

%\begin{proof}
%Since $\Sigma$ is diagonal, the eigenvectors of $\Sigma$ are exactly the standard basis vectors $e_1,\dots,e_{k-1}$.  
%For rank-$1$ PCA, the optimal reconstruction operator $H_1$ minimizes
%\[
%\rho(X, H_1 X) = \| X - H_1 X \|_F^2 = \sum_{j=1}^{k-1} \| X_j - (H_1 X)_j \|_2^2.
%\]
%This is minimized by choosing $H_1 = e_{j^*} e_{j^*}^\top$, where $j^* = \arg\max_j \| X_j \|_2^2 = \arg\max_j \operatorname{Var}(X_j)$.  

%For rank-$r$ PCA, the optimal reconstruction operator is
%\[
%H_r = \sum_{s=1}^r e_{j_s} e_{j_s}^\top,
%\]
%where $j_1,\dots,j_r$ correspond to the $r$ largest diagonal entries of $\Sigma$. By construction, $H_r X$ preserves exactly these predictors and zeroes out all others.  

%In terms of regression operators, $L_\ell$ selects the first $\ell$ predictors, or more generally the $\ell$ chosen coordinates. In the diagonal covariance case, by aligning the ordering of $\ell$ with the $r$ largest eigenvalues, the PCA operator $H_r$ and the regression operator $A \in L_\ell$ act identically on $X$. Hence, coincidence holds.
%\end{proof}

%This proposition formalizes the intuitive idea, that regression-type variable selection and PCA-type reconstruction coincide exactly when the predictors are uncorrelated and the selected PCA subspace aligns with coordinate axes. In the general case of correlated predictors, PCA spans rotated subspaces and the two procedures generally diverge.

%We now turn to concrete examples in which the two approaches either coincide or diverge. This serves two purposes: (i) it illustrates how the abstract operator classes $L_\ell$ (regression) and $\mathcal{H}_r$ (PCA) behave in practice, and (ii) it highlights the role of covariance structure, signal alignment, and distributional assumptions in determining when the two procedures can agree.

%\paragraph{Orthogonal Predictors as a Trivial Semiring Case}

%Suppose $X_1, \dots, X_p \sim \mathcal{N}(0,1)$ independent, and let $Y = X_1 + \varepsilon$, $\varepsilon \sim \mathcal{N}(0,1)$. Here the covariance matrix is diagonal, so PCA selects the coordinate axes as eigenvectors. Linear regression selects $X_1$ as the relevant predictor. Thus, PCA and regression coincide, and both correspond to a semiring element $A \in L_k \cap \mathcal{H}_r$ that projects onto $X_1$. This example illustrates the trivial intersection of the linear regression semiring and the PCA semiring (since only one projector semigroup works as a semiring), both are aligned with the coordinate basis.

%\paragraph{Correlated Predictors as a Contradiction}

%Now let $X_2 = X_1 + \eta$ with $\eta \sim \mathcal{N}(0,0.01)$, and again $Y = X_1 + \varepsilon$. Then regression identifies $X_1$ as the unique source of signal, but PCA selects approximately $X_1 + X_2$, the direction of maximal variance. Algebraically, the operator $A_\beta$ encoding the linear regression functional lies outside the image of the PCA projection $\mathcal{H}_1$. Thus, the semiring elements representing PCA and linear regression diverge. This example demonstrates that the two structures, though unified algebraically, may generate incompatible models.

%\paragraph{Loss of Signal under Projection}

%Let $X = (X_1,X_2)$ with $\mathrm{Var}(X_1)\gg \mathrm{Var}(X_2)$ and $\beta=(0,1)$. Then PCA selects $X_1$ as first component, but the signal lies in $X_2$. In operator terms, $A_\beta$ is orthogonal to $\mathrm{Im}(A_X)$. Thus the regression functional is annihilated by the PCA projection. This illustrates how projection may erase inferentially relevant information, that cannot be captured by variance maximization of PCA alone.

%\paragraph{Heavy-tailed Distributions}

%If $X \sim S(\alpha)$ is $\alpha$-stable with $\alpha < 2$, the covariance matrix does not exist, and PCA cannot be defined in its classical form. Regression, however, can still be expressed through conditional expectation. This shows that the PCA semiring $\mathcal{H}_r$ is undefined, while the regression semiring $L_\ell$ remains meaningful. Hence, the semiring framework distinguishes between procedures that depend on second-order structure (PCA) and those that do not (regression). This provides a structural explanation for why PCA is not universally applicable.

% Kapitel 6
\section{Novel Contributions from the Semiring Perspective}

The algebraic framework developed in this paper does not merely unify linear regression and PCA in a common operator language. It also to design several novel conceptual tools for statistical modeling, each highlighting different aspects of coincidence and contradiction between the two methods. 
We describe three aspects below.

\subsection{Non-commutativity as a diagnostic measure}

Within the semiring of structured operators, the composition of two modeling procedures is in general non-commutative: applying regression after PCA differs from applying PCA after linear regression. This lack of commutativity can itself be used as a \emph{diagnostic tool}. Within the semiring of structured operators $\mathcal{A} \subseteq P$, consider two operators $A, B \in \mathcal{A}$ representing, for instance, regression-type and PCA-type transformations. The \emph{degree of contradiction} quantifies the extent to which these two operators fail to commute:

\begin{defn}[Degree of Contradiction]
Let $\|\cdot\|_F$ denote the Frobenius norm. The degree of contradiction between $A$ and $B$ is defined as
\[
\Delta(A,B) := \| AB - BA \|_F,
\]
interpreted as a Euclidean proxy for the underlying model-level divergence.
\end{defn}

\paragraph{Interpretation.}
\begin{itemize}
    \item In the semiring context, $\mathcal{A}$ is equipped with an additive operation $\oplus$ 
          and a multiplicative operation $\otimes$ (matrix multiplication). 
          Commutativity under $\otimes$ is not guaranteed in general. 
    \item $\Delta(A,B) = 0$ implies $AB = BA$, i.e., the two operators \emph{commute} within the semiring. 
          In this case, there exists a well-defined joint action of $A$ and $B$, and a unique optimal operator 
          $H^* \in \mathcal{A}$ may be identified that simultaneously satisfies both objectives.
    \item $\Delta(A,B) > 0$ indicates a \emph{structural contradiction} between the operators: 
          their actions on some subspace of the predictors are incompatible. 
          Algebraically, there exists no single element in $\mathcal{A}$ that preserves both transformations exactly.
\end{itemize}

\paragraph{Implication for Joint Estimation.}
Formally, consider the problem of finding a joint optimal operator
\[
H^* := \arg\min_{H \in \mathcal{A}} \left\{ \rho_\mathrm{reg}(H) + \rho_\mathrm{PCA}(H) \right\},
\]
where $\rho_\mathrm{reg}$ and $\rho_\mathrm{PCA}$ denote the model-level loss functionals corresponding to regression and PCA, respectively. 
\begin{itemize}
    \item If $\Delta(A,B) = 0$, then the minimization above admits a clear solution 
          $H^* = AB = BA$; the semiring structure guarantees closure under $\otimes$, and the joint loss is minimized by the common operator.
    \item If $\Delta(A,B) > 0$, no such element $H^*$ exists that simultaneously minimizes both losses exactly. 
          The optimization must instead select a compromise operator, e.g., an \emph{argmin} over a combined loss, reflecting the algebraic incompatibility.
\end{itemize}

This measure turns the algebraic structure into a quantitative measure to assess \emph{ex ante} whether PCA and linear regression are likely to produce coherent or contradictory conclusions on a given dataset. 

\subsection{Sufficiency Gap in the Semiring Framework}

In classical statistics, regression and PCA rely on different sufficient statistics:
\[
T_{\text{reg}}(X,Y) = (X^\top X, X^\top Y), \qquad
T_{\text{PCA}}(X) = X^\top X.
\]
Regression retains all information about the parameter $\beta$, while PCA retains only information about the variance structure of $X$. The concept of sufficiency gaps explains, why PCA cannot generally substitute for regression. The idea is, that the sufficient statistic it relies on omits all information about the response variable $Y$. 

\paragraph{Semiring perspective.}  
Let $P$ denote the semiring of structured operators, with subsets
$L_\ell \subset P$ representing regression operators and
$\mathcal{H}_r \subset P$ representing PCA operators. 
Each operator $A \in P$ acts on the data matrix $X$ and extracts a subspace
\(\mathrm{Im}(A) \subseteq \mathbb{R}^{k-1}\). 
We can interpret sufficiency in terms of these subspaces: an operator is sufficient for a parameter if the parameter is a functional of the projected data.

\begin{defn}[Sufficiency Gap]  
Let $A_{\text{reg}} \in L_\ell$ and $A_{\text{PCA}} \in \mathcal{H}_r$. 
We define the \emph{sufficiency gap} of PCA relative to regression as the dimension of the missing subspace:
\[
\text{Gap}(A_{\text{PCA}} \parallel A_{\text{reg}}) 
:= \dim \Big( \mathrm{Im}(A_{\text{reg}}) \;\setminus\; \mathrm{Im}(A_{\text{PCA}}) \Big) = I(\beta; T_{\text{reg}}) - I(\beta; T_{\text{PCA}}),,
\]
where $\mathrm{Im}(A)$ denotes the image of $A$.  
Equivalently, one can consider the Frobenius-norm of the difference between the projections:
\[
\text{Gap}(A_{\text{PCA}} \parallel A_{\text{reg}}) 
\approx \| A_{\text{reg}} - A_{\text{PCA}} A_{\text{reg}} \|_F.
\]
\end{defn}

\paragraph{Interpretation.}  
\begin{itemize}
    \item $\text{Gap} = 0$: PCA preserves all subspaces relevant for regression; operators are fully compatible.
    \item $\text{Gap} > 0$: PCA discards information relevant for regression; the operators diverge in the semiring sense.
\end{itemize}

\paragraph{Connection to semiring structure.}  
The semiring operations of addition and multiplication allow us to analyze combined procedures:
\begin{itemize}
    \item Addition $A + B$ combines subspaces (union of images) of two operators.
    \item Multiplication $AB$ corresponds to sequential application; non-commutativity $AB \neq BA$ quantifies incompatibility.
\end{itemize}
The sufficiency gap formalizes this incompatibility in terms of lost regression-relevant subspaces, making it a semiring-consistent measure of divergence between PCA and regression.
\subsection{Contradiction maps in operator space}

We further describe regions of operator-parameter space where regression- and 
PCA-type procedures coincide or diverge. Let $A \in P$ encode a Gaussian model
\[
Y = X \beta + \varepsilon, \qquad X \sim \mathcal{N}(0, \Sigma),
\]
with regression part $A_\beta$ and PCA projection part $A_\mu$.

Define the \emph{coincidence set} of operators
\[
\mathcal{C}_r := \{ A \in P : \operatorname{Im}(A_\beta) \subseteq \operatorname{Im}(A_\mu), \; \operatorname{rank}(A_\mu) = r \},
\]
i.e., the set of operators where the regression signal lies entirely within the PCA-projected subspace. In this region, regression and PCA-based reconstruction yield identical fitted values for the predictors.

The complement
\[
\mathcal{C}_r^c := P \setminus \mathcal{C}_r
\]
is the \emph{contradiction set}, where PCA projections discard part of the regression signal encoded in $A_\beta$.

\paragraph{Interpretation.}  
Visualizing $\mathcal{C}_r$ and $\mathcal{C}_r^c$ as subsets of the semiring $P$ produces \emph{contradiction maps}: geometric representations of where and why regression and PCA disagree in operator space.

% Kapitel 7
\section{Model Extension and Reduction: A McCullagh-Inspired View}

The semiring operator framework provides a natural formalization of the model extension and reduction concepts emphasized by McCullagh \cite{mccullagh}. By treating regression and PCA as operators acting on the model space rather than on finite data, we can define these processes consistently at the distributional level, independently of any particular dataset.

\subsection{Illustrations of Model Extendability and Operator Structure}

McCullagh emphasizes that a statistical model is meaningful only if it admits natural extension, reduction, and invariance properties with respect to the inferential universe of interest. Using the operator class, we can reinterpret classical procedures such as PCA and linear regression in this light and highlight structural differences that are invisible at the data level. The following examples illustrate key phenomena.

\paragraph{Example 1: Extendability and the non-natural PCA projector}

\textbf{Phenomenon.}
Inference requires that a model admit a natural extension to larger datasets, populations, or future units. PCA based on a finite sample may fail this criterion if the empirical principal subspace is highly sample-dependent.


\textbf{Formal description.}
Let $\tilde X^{(n)} \in \mathbb{R}^{k \times n}$ denote a dataset with $n$ units, and let $H^{(n)} = Q_r^{(n)} (Q_r^{(n)})^\top$ be the empirical rank-$r$ PCA projector. If the mapping $n \mapsto H^{(n)}$
does not converge (in probability or almost surely) to a population-level operator $H^{(\infty)}$ intrinsic to the model, then the reduction is not natural in McCullagh’s sense. Predictions or inferential statements using $H^{(n)}$ cannot be canonically extended to new units.


\textbf{Consequence.}
The semiring viewpoint makes this explicit: PCA operators form a semigroup $\mathcal{H}_r$, but only functorial extendability guarantees valid inference. Lack of this property signals that PCA is primarily descriptive.


\paragraph{Example 2: Natural parameters and coordinate transformations}

Natural parameters should transform equivariantly under relevant experiment morphisms. In the operator framework, this corresponds to equivariance under coordinate permutations or other symmetries of the predictor space.

\textbf{Formal description.}
Let $P$ be a permutation matrix on predictors and $A \in P$ an operator in the semiring. Define the conjugated operator
\[
A^\Pi = \begin{bmatrix} 1 & 0 \\ 0 & P \end{bmatrix} A \begin{bmatrix} 1 & 0 \\ 0 & P^\top \end{bmatrix}.
\]
A parameter $\theta(A)$ is natural if $\theta(A^\Pi) = \Pi \cdot \theta(A),$ i.e., it transforms consistently with the permutation.

\textbf{Implication.}
Regression coefficients satisfy this property for coordinate relabelings, so $L_\ell$ admits a natural parameterization. PCA eigenvectors, however, are only defined up to sign and scaling, and may fail to transform naturally under general morphisms. The semiring framework precisely characterizes which transformations preserve which operator families.

\paragraph{Example 3: Extensive vs. intensive aggregation and commutativity}

\textbf{Phenomenon.}
Operators should interact appropriately with aggregation over units (summing or averaging). This distinguishes extensive from intensive inferential properties.

\textbf{Algebraic test.}
Let $G_n : \mathbb{R}^{k \times n} \to \mathbb{R}^k$ denote aggregation (e.g., unit-wise mean). We ask whether there exists $A'$ in the same semiring such that
$G_n(A \tilde X^{(n)}) = A' \, G_n(\tilde X^{(n)}) \quad \forall \tilde X^{(n)}.$
If so, extraction and aggregation commute; otherwise, the reduction is aggregation-sensitive.

\textbf{Implication.}
PCA is sensitive to whether units are aggregated before or after projection, while regression estimators based on $(X^\top X, X^\top Y)$ commute naturally with aggregation.

\paragraph{Example 4: Privacy/protection as an algebraic constraint}

Projection operators can be interpreted as privacy-preserving transformations, which restrict the amount of recoverable information.

\textbf{Algebraic observation.}
Define $\mathcal{P}r = \{ A \in P : A_\sigma = 0, \; \operatorname{rank}(A_\mu) \le r \}.$
Mapping $\tilde X \mapsto A \tilde X$ for $A \in \mathcal{P}_r$ preserves privacy by limiting reconstructable components. However, if the regression signal lies in the nullspace of $A_\mu$, identifiability is lost. The semiring formalism expresses this trade-off algebraically: privacy = membership in $\mathcal{P}_r$, identifiability = non-membership in the annihilator of $\mathcal{P}_r$.

\paragraph{Example 5: Models beyond second-order structure}

Relying on moments is implicit in many procedures. PCA depends on covariances (second-order structure), whereas regression via conditional expectations can extend beyond second moments.

\textbf{Algebraic consequence.}
For heavy-tailed distributions (e.g., $\alpha$-stable with $\alpha < 2$), the PCA semigroup $\mathcal{H}_r$ is ill-defined, while regression operators $L_\ell$ can remain meaningful under robust or tail-sensitive formulations. The semiring framework separates second-order-dependent families from more general operator families, exposing structural limitations of PCA.

These examples illustrate core McCullagh principles: extendability, naturalness under morphisms, commutativity with aggregation, privacy constraints, and reliance on moment assumptions. By expressing them in semiring/operator terms, we make explicit which statistical procedures admit canonical extensions and which are only descriptive. The framework therefore provides a systematic language for evaluating the inferential suitability of linear regression, PCA, and their hybrids.


\subsection{Model Extension: From Small to Large}

Consider a family of centered Gaussian models for the predictors:
\[
X \sim \mathcal{N}(0, \Sigma), \qquad \Sigma \in \mathcal{S}_{+}^{k}.
\]

A smaller model corresponds to restricting attention to a subspace of the full predictor space. Formally, let $A_\mu^{(\ell)} \in P$ be a rank-$\ell$ operator selecting an $\ell$-dimensional subspace. The associated model family is
\[
\mathcal{P}\ell := \bigl\{ \mathcal{N}(0, A_\mu^{(\ell)} \, \Sigma \, (A_\mu^{(\ell)})^\top) : A_\mu^{(\ell)} \text{ is a rank-}\ell \text{ projection} \bigr\}.
\]

Extending to a larger model means embedding this operator into a higher-dimensional space while preserving the distributional structure:
\[
A_\mu^{(\ell)} \mapsto A_\mu^{(\ell’)} \in P, \quad \ell’ > \ell, \quad
\mathcal{P}_\ell \subset \mathcal{P}_{\ell’} \subset \mathcal{P}.
\]

Here, the extension is defined at the operator-level. It enlarges the subspace and parameterization without reference to any particular finite sample. This formalizes McCullagh’s principle that enlarging a model should be a distributionally consistent operation, not an artifact of data.

\subsection{Model Reduction: From Large to Small}

Reduction corresponds to projecting a larger model onto a lower-dimensional subspace while preserving the family’s algebraic structure. Let $A \in P$ denote a structured operator in block form:
\[
A = \begin{bmatrix} A_\sigma & A_\beta \\ 0 & A_\mu \mathbb{1}_{(k-1) \times (k-1)}\end{bmatrix}.
\]

For regression operators $A \in L_\ell$, $A_\mu$ restricts the predictors to a coordinate subspace; for PCA operators $A \in \mathcal{H}_r$, $A_\mu$ restricts the predictors to an eigensubspace of $\Sigma$. The transformed model is
$X \mapsto AX \sim \mathcal{N}\bigl(0, A \Sigma A^\top \bigr),$
which preserves the Gaussian family while compressing the parameter space.

Thus, reduction is a model-level operation: it replaces $\Sigma$ with $A \Sigma A^\top$ while retaining all algebraic relations. It is not defined in terms of a finite data realization but only in terms of the distribution and the chosen operator.

\subsection{Data vs. Model: Estimation as a Separate Layer}

Empirical datasets provide estimates of model parameters, e.g.,
\[
\hat\Sigma_n = \frac{1}{n} \sum_{i=1}^n X_i X_i^\top \to \Sigma \text{ as } n \to \infty.
\]
Applying PCA to $\hat\Sigma_n$ yields an empirical operator approximating the model-level PCA projection $Q_r Q_r^\top$. Importantly:

\begin{itemize}
\item The operator itself is defined at the model level and does not depend on any specific sample.
\item More data improves the accuracy of estimation, but does not change the underlying operator or the model family.
\end{itemize}

This distinction resolves the classical “data vs. more data” paradox highlighted by McCullagh: models are defined independently of finite data, and larger samples refine only the estimates of their parameters, not the family itself.

\subsection{Consequences for Unified Modeling}

The semiring framework naturally accommodates both extension (embedding operators into larger subspaces) and reduction (projecting via $A \Sigma A^\top$) at the level of distributions.

\begin{itemize}
\item Regression and PCA operators can be interpreted as structured reductions or extensions, satisfying McCullagh’s requirement of distributional consistency.
\item Data-level procedures (e.g., PCA on $\hat\Sigma_n$) can be understood as approximations of model-level operations.
\item This unifies both classical variable selection and dimensionality reduction within a semiring-based algebra, where operations act on models rather than on finite datasets.
\end{itemize}

In this sense, the semiring perspective provides a model-first formalization of McCullagh’s principles, separating statistical modeling from empirical estimation and ensuring consistency under both extension and reduction.

\newpage

% Kapitel 10
\section{Ideen}

\subsection{Chapter 6}

\subsubsection{Semiring Factorization of Inference Goals}

A key insight from the semiring perspective is that regression and PCA differ not in the underlying data model, but in their \emph{inferential goals}: regression targets conditional relationships, whereas PCA captures marginal structure.  
Our operator-semiring framework makes this distinction algebraically explicit: one can represent target quantities themselves as elements or ideals within the semiring.  

\begin{itemize}
    \item \textbf{Predictive vs. descriptive vs. protective goals:} The algebraic classification allows us to categorize inference objectives according to whether they aim at prediction, descriptive dimensionality reduction, or information protection.
    \item \textbf{Generalization to other methods:} Techniques such as Lasso, Ridge regression, or Canonical Correlation Analysis can be interpreted as specialized ideals or subsemirings within the same overarching operator framework, making apparent their structural relations to classical PCA and regression.
\end{itemize}

This factorization offers a unifying algebraic lens to compare and contrast diverse inferential targets within a single semiring structure.

\subsubsection{Robust Semiring Extensions}

Classical PCA is sensitive to heavy-tailed distributions (\(\alpha\)-stable with \(\alpha<2\)), whereas regression can remain well-defined.  
The semiring framework suggests a path to \emph{robust operator structures}:

\begin{itemize}
    \item Construct operator-semiring variants based on robust statistics, e.g., medians, median absolute deviations, or quantile-based projections.
    \item This yields an algebraic framework that unifies classical and robust procedures, preserving closure properties and semiring operations.
    \item Practical benefit: PCA-like dimension reduction and regression-based inference become robust to outliers and non-Gaussian data without leaving the semiring formalism.
\end{itemize}

\subsubsection{Universal Approximation via Operator Semirings}

A natural question is: which set of operators suffices to represent all meaningful inferential tasks, such as projections, regressions, or classifications, within this algebraic framework?  

\begin{itemize}
    \item \textbf{Universal representation:} Analogous to universal approximation theorems in analysis, one can envision a semiring-based universality result where PCA and regression operators serve as a \emph{basis set} from which complex inference procedures can be constructed.
    \item This perspective characterizes classical dimension reduction and predictive models as fundamental building blocks of a more general operator-algebraic inference system.
\end{itemize}

\subsubsection{Dynamic Semiring Evolution}

In real-world datasets, PCA subspaces and regression coefficients can change with sample growth or temporal evolution. The semiring formalism can be extended to capture such dynamics:

\begin{itemize}
    \item Define a \emph{time-dependent semiring} in which operators evolve continuously or discretely, forming a flow in a non-commutative algebraic space.
    \item Applications include online learning, streaming data analysis, and adaptive modeling, where both predictive and descriptive components of the model update coherently within the semiring.
\end{itemize}

This dynamic viewpoint integrates statistical adaptation into the algebraic structure, allowing the framework to reflect not only static inferential relationships but also their temporal evolution.


\subsection{Chapter 4: Sufficiency in regression- and PCA-type operator models}
\label{subsec:sufficiency}

The algebraic framework developed so far allows us to reinterpret sufficiency of statistics in terms of structured operators. Recall that for a Gaussian model, the empirical mean and covariance are sufficient for $(\mu,\Sigma)$. Within our semiring setting, sufficiency acquires a finer meaning: each operator $A \in P$ extracts exactly that part of the distribution of $\tilde{X}$ which is relevant for the parameters encoded in $A$.

\paragraph{Setup.}  
Let $\tilde{X} \sim \mathcal{N}(0, \Sigma)$ with $\Sigma \in \mathbb{R}^{k \times k}$ positive definite, and let $A \in P$. Define
\[
Y = T_A(\tilde{X}) := A \tilde{X}.
\]
Then
\[
Y \sim \mathcal{N}(0, A \Sigma A^\top).
\]
By the factorization theorem, $T_A(\tilde{X})$ is a sufficient statistic for any parameter of the Gaussian distribution that can be expressed as a functional of $A \Sigma A^\top$. In other words, $A$ defines a sufficient reduction of the data.

\paragraph{Regression-type operators.}  
For $A \in L_\ell$, the first row of $A$ encodes regression coefficients $\beta$ acting only on predictors $X_1,\dots,X_\ell$. The likelihood depends on $\beta$ solely through
\[
(X_1, \dots, X_\ell)^\top.
\]
Hence $T_A(\tilde{X})$ is sufficient for $\beta$ restricted to this subset. Algebraically, sufficiency coincides with the coordinate projection structure of $L_\ell$.

\paragraph{PCA-type operators.}  
For $A \in \mathcal{H}_r$, the block $A_\mu$ is a projection of rank $r$. The transformed variable $T_A(\tilde{X}) = A \tilde{X}$ depends on $\Sigma$ only through
\[
A_\mu \Sigma A_\mu^\top,
\]
which contains all information about the subspace preserved by $A_\mu$. Thus $T_A(\tilde{X})$ is sufficient for the projected covariance structure, i.e., for the eigenspaces corresponding to the chosen low-rank reconstruction, while discarding orthogonal directions.

\paragraph{Objective.}  
Show that \(T_A(\tilde{X})\) is a sufficient statistic for parameters of interest in a model class \(\mathcal{M}\) defined via a subset of the covariance structure of \(\tilde{X}\).

\paragraph{Likelihood-based Derivation.}  
The density of \(\tilde{X}\) is
\[
f_{\tilde{X}}(\tilde{x}; \Sigma) = \frac{1}{(2\pi)^{k/2} |\Sigma|^{1/2}} \exp\Big(-\frac{1}{2} \tilde{x}^\top \Sigma^{-1} \tilde{x}\Big), \quad \tilde{x} \in \mathbb{R}^k.
\]

Consider the transformation \(Y = A \tilde{X}\). Using the change-of-variables formula (valid for full-rank or rank-preserving \(A\)):
\[
f_Y(y; \Sigma) = \frac{1}{(2\pi)^{k/2} |A \Sigma A^\top|^{1/2}} \exp\Big(-\frac{1}{2} y^\top (A \Sigma A^\top)^{-1} y\Big), \quad y \in \operatorname{Im}(A).
\]

Notice that the density depends on \(y\) only through the quadratic form
\[
y^\top (A \Sigma A^\top)^{-1} y,
\]
i.e., on \(T_A(\tilde{X}) = A \tilde{X}\). By the factorization theorem, this implies $T_A(\tilde{X}) = A \tilde{X}$ is sufficient for all parameters encoded in $A$.

\paragraph{Comparison.}  
\begin{itemize}
    \item In regression ($L_\ell$), sufficiency is coordinate-aligned: only the selected predictors matter, and $T_A(\tilde{X})$ compresses the data to those coordinates without losing information about $\beta$.
    \item In PCA ($\mathcal{H}_r$), sufficiency is subspace-aligned: the sufficient statistic is the projection of $\tilde{X}$ onto a low-dimensional subspace, retaining all information about that subspace but losing coordinate-level interpretability.
\end{itemize}

\paragraph{Unified semiring perspective.}
Both regression and PCA thus admit sufficient statistics in the Gaussian model, but with different invariances: regression is invariant under scaling within selected coordinates, PCA under rotations within the chosen subspace. In the semiring language, this corresponds to sufficiency being stable under $L_\ell$ for regression and under $\mathcal{H}_r$ for PCA. The divergence between the two models therefore arises not from the lack of sufficiency, but from the fact that different operator families define different sufficient reductions of the same underlying distribution.

% Kapitel 7
\section{Comparison with Existing Selection Models}

%6. Comparison with existing selection models
%	•	Forward selection
%	•	backward selection
%	•	Lasso and Ridge

To better understand the divergence, we contrast PCA selection with other 
classical selection procedures. 

\subsection{Forward Selection}
Forward selection chooses predictors by stepwise reduction in residual sum of squares. 
It may include variables with small variance if they are highly correlated with $Y$. 
In contrast, PCA ignores $Y$ and selects only based on variance.

\subsection{Lasso and Ridge}
Lasso solves
\[
\hat\beta = \arg\min_\beta \|Y - X\beta\|^2 + \lambda \|\beta\|_1,
\]
producing sparse solutions aligned with $Y$. Ridge regression shrinks coefficients but does not select variables. 
Both differ fundamentally from PCA: Lasso depends on correlation with $Y$, PCA does not. 
Interestingly, ridge regression directions often resemble PCA directions if variance dominates, but objectives are different.

\subsection{Examples and Contradictions}

We illustrate the theoretical results with explicit examples that show the divergence 
between PCA- and regression-based selection.

\begin{example}[Coincidence under Orthogonality]
Let $X_1, X_2 \sim \mathcal{N}(0,1)$ independent, and define $Y = X_1 + \varepsilon$. 
The covariance matrix is diagonal, so PCA directions coincide with the coordinate axes. 
Both PCA and regression select $X_1$, and the procedures coincide. 
\end{example}

\begin{example}[Contradiction with Correlated Predictors]
Let $X_1 \sim \mathcal{N}(0,1)$ and $X_2 = X_1 + \eta$ with $\eta \sim \mathcal{N}(0,0.01)$, 
and define $Y = X_1 + \varepsilon$, $\varepsilon \sim \mathcal{N}(0,1)$. 
Then $X_1$ carries the true signal. 
However, PCA selects approximately the direction $X_1+X_2$ (largest variance), 
whereas regression identifies $X_1$ as the relevant predictor. 
Thus PCA-based variable selection and regression-based selection diverge. 
\end{example}

\begin{example}[Loss of Signal by Projection]
Let $X = (X_1,X_2)$, $\beta = (0,1)$, and suppose PCA selects only $X_1$ 
because $\mathrm{Var}(X_1) \gg \mathrm{Var}(X_2)$. 
Then PCA regression discards the true signal $X_2$, and the fitted model becomes $Y=\varepsilon$, 
i.e.\ pure noise. This illustrates the structural contradiction: 
projection may eliminate the signal entirely. 
\end{example}

\begin{remark}
These examples can be rephrased algebraically: 
if $\beta \perp \mathrm{Im}(A_X)$ for a selection operator $A$, 
then $AX\beta = 0$ and the resulting regression reduces to noise. 
This makes explicit the conflict between PCA projection and regression selection. 
\end{remark}


% Kapitel 8
\section{PCA-Regression}

%7. PCA-Regression

\subsection{PCA Regression and Generalized Models}

Allowing off-diagonal structure in $A_X$, that is, nonzero entries beyond the diagonal, enables transformations of the predictors prior to regression. This yields a class of models commonly referred to as \emph{principal component regression} (PCR).

\paragraph{Definition (PCA Regression).}
Let $V \in \mathbb{R}^{(k-1) \times r}$ be a matrix of orthonormal eigenvectors of the empirical covariance matrix of $X$, corresponding to the $r$ largest eigenvalues. Define transformed predictors:
\[
Z := V^\top X.
\]
A regression is then performed on the transformed variables:
\[
Y = \alpha_0 + \alpha^\top Z + \varepsilon = \alpha_0 + \alpha^\top V^\top X + \varepsilon.
\]
This corresponds to a transformation matrix $A \in \mathcal{A}$ of the form:
\[
A = 
\begin{bmatrix}
A_\sigma & A_\beta \\
0 & A_X
\end{bmatrix}, \quad \text{with } A_X = V^\top.
\]
The first row of $A$ performs a regression on the principal components; the remaining rows may perform identity operations, further transformations, or be omitted entirely.

\paragraph{Interpretation.}
PCA regression performs variable selection and regularization implicitly by discarding low-variance components in $X$. Compared to standard regression:
\begin{itemize}
    \item PCA regression avoids multicollinearity by orthogonalizing the input space,
    \item it introduces bias by truncating directions with small variance, which may still be predictive.
\end{itemize}



\paragraph{Example.}
Let $X \in \mathbb{R}^{n \times p}$, and compute the eigenvalue decomposition:
\[
\frac{1}{n} X^\top X = Q \Lambda Q^\top.
\]
Choose the top $r$ eigenvectors $V = Q_{[:,1:r]}$, and define:
\[
Z = X V.
\]
Then PCR fits the model:
\[
Y = Z \alpha + \varepsilon = X V \alpha + \varepsilon,
\]
which can be interpreted as linear regression with restricted parameter space $\beta = V \alpha$.

\paragraph{Relation to Generalized Models.}
Allowing more flexible structures in $A_X$ (e.g., non-orthogonal, sparse, or structured matrices) leads to hybrid models:
\begin{itemize}
    \item Sparse PCA regression
    \item Supervised PCA (SPCA)
    \item Partial least squares (PLS)
\end{itemize}

All these models can be written in the general form $Y = A \tilde{X}$ with specific choices of $A \in \mathcal{A}$, where $A_X$ encodes dimension reduction or selection, and $A_\beta$ encodes regression.

% Kapitel 9
\section{Outlook}

\paragraph{Conclusion.}
PCA regression illustrates how dimensionality reduction and prediction can be jointly modeled through structured linear operators. It fits naturally into our unified matrix-based framework, and serves as a concrete example where variance-preserving transformations and regression estimation coexist within the same algebraic structure.

\subsection{Future Work}
\begin{itemize}
    \item Define and use variation as in Schlather \& Reinbott.
    \item Explore orthogonality of selected variables via scalar products.
    \item Analyze algebraic closures and group-like extensions (e.g., quasi-inverses).
    \item Extending the coincidence conditions to heavy-tailed or stable distributions 
    where classical variance is undefined. 
    \item An algebraic classification of operator subsets (subsemirings, ideals) 
    and their statistical interpretation. 
    \item Simulation studies beyond the Gaussian case to illustrate 
    the persistence of contradictions. 
\end{itemize}

\subsection{Conclusion}
We provide a theoretical framework that explains why PCA and regression differ, despite acting on similar statistical data. The structure introduced here enables PCA to be seen as a constrained case of statistical estimation — specifically, one where directionality is relaxed. This algebraic treatment lays the foundation for generalized PCA and structured variable selection models.


\bibliographystyle{plain}
\bibliography{bibliography}

\end{document}