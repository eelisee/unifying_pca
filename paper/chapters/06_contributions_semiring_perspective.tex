\section{Novel Contributions from the Semiring Perspective}

The algebraic framework developed in this paper does not merely unify linear regression and PCA in a common operator language. It also to design several novel conceptual tools for statistical modeling, each highlighting different aspects of coincidence and contradiction between the two methods. 
We describe three aspects below.

\subsection{Non-commutativity as a diagnostic measure}

Within the semiring of structured operators, the composition of two modeling procedures is in general non-commutative: applying regression after PCA differs from applying PCA after linear regression. This lack of commutativity can itself be used as a \emph{diagnostic tool}. Within the semiring of structured operators $\mathcal{A} \subseteq P$, consider two operators $A, B \in \mathcal{A}$ representing, for instance, regression-type and PCA-type transformations. The \emph{degree of contradiction} quantifies the extent to which these two operators fail to commute:

\begin{defn}[Degree of Contradiction]
Let $\|\cdot\|_F$ denote the Frobenius norm. The degree of contradiction between $A$ and $B$ is defined as
\[
\Delta(A,B) := \| AB - BA \|_F,
\]
interpreted as a Euclidean proxy for the underlying model-level divergence.
\end{defn}

\paragraph{Interpretation.}
\begin{itemize}
    \item In the semiring context, $\mathcal{A}$ is equipped with an additive operation $\oplus$ 
          and a multiplicative operation $\otimes$ (matrix multiplication). 
          Commutativity under $\otimes$ is not guaranteed in general. 
    \item $\Delta(A,B) = 0$ implies $AB = BA$, i.e., the two operators \emph{commute} within the semiring. 
          In this case, there exists a well-defined joint action of $A$ and $B$, and a unique optimal operator 
          $H^* \in \mathcal{A}$ may be identified that simultaneously satisfies both objectives.
    \item $\Delta(A,B) > 0$ indicates a \emph{structural contradiction} between the operators: 
          their actions on some subspace of the predictors are incompatible. 
          Algebraically, there exists no single element in $\mathcal{A}$ that preserves both transformations exactly.
\end{itemize}

\paragraph{Implication for Joint Estimation.}
Formally, consider the problem of finding a joint optimal operator
\[
H^* := \arg\min_{H \in \mathcal{A}} \left\{ \rho_\mathrm{reg}(H) + \rho_\mathrm{PCA}(H) \right\},
\]
where $\rho_\mathrm{reg}$ and $\rho_\mathrm{PCA}$ denote the model-level loss functionals corresponding to regression and PCA, respectively. 
\begin{itemize}
    \item If $\Delta(A,B) = 0$, then the minimization above admits a clear solution 
          $H^* = AB = BA$; the semiring structure guarantees closure under $\otimes$, and the joint loss is minimized by the common operator.
    \item If $\Delta(A,B) > 0$, no such element $H^*$ exists that simultaneously minimizes both losses exactly. 
          The optimization must instead select a compromise operator, e.g., an \emph{argmin} over a combined loss, reflecting the algebraic incompatibility.
\end{itemize}

This measure turns the algebraic structure into a quantitative measure to assess \emph{ex ante} whether PCA and linear regression are likely to produce coherent or contradictory conclusions on a given dataset. 

\subsection{Sufficiency Gap in the Semiring Framework}

In classical statistics, regression and PCA rely on different sufficient statistics:
\[
T_{\text{reg}}(X,Y) = (X^\top X, X^\top Y), \qquad
T_{\text{PCA}}(X) = X^\top X.
\]
Regression retains all information about the parameter $\beta$, while PCA retains only information about the variance structure of $X$. The concept of sufficiency gaps explains, why PCA cannot generally substitute for regression. The idea is, that the sufficient statistic it relies on omits all information about the response variable $Y$. 

\paragraph{Semiring perspective.}  
Let $P$ denote the semiring of structured operators, with subsets
$L_\ell \subset P$ representing regression operators and
$\mathcal{H}_r \subset P$ representing PCA operators. 
Each operator $A \in P$ acts on the data matrix $X$ and extracts a subspace
\(\mathrm{Im}(A) \subseteq \mathbb{R}^{k-1}\). 
We can interpret sufficiency in terms of these subspaces: an operator is sufficient for a parameter if the parameter is a functional of the projected data.

\begin{defn}[Sufficiency Gap]  
Let $A_{\text{reg}} \in L_\ell$ and $A_{\text{PCA}} \in \mathcal{H}_r$. 
We define the \emph{sufficiency gap} of PCA relative to regression as the dimension of the missing subspace:
\[
\text{Gap}(A_{\text{PCA}} \parallel A_{\text{reg}}) 
:= \dim \Big( \mathrm{Im}(A_{\text{reg}}) \;\setminus\; \mathrm{Im}(A_{\text{PCA}}) \Big) = I(\beta; T_{\text{reg}}) - I(\beta; T_{\text{PCA}}),,
\]
where $\mathrm{Im}(A)$ denotes the image of $A$.  
Equivalently, one can consider the Frobenius-norm of the difference between the projections:
\[
\text{Gap}(A_{\text{PCA}} \parallel A_{\text{reg}}) 
\approx \| A_{\text{reg}} - A_{\text{PCA}} A_{\text{reg}} \|_F.
\]
\end{defn}

\paragraph{Interpretation.}  
\begin{itemize}
    \item $\text{Gap} = 0$: PCA preserves all subspaces relevant for regression; operators are fully compatible.
    \item $\text{Gap} > 0$: PCA discards information relevant for regression; the operators diverge in the semiring sense.
\end{itemize}

\paragraph{Connection to semiring structure.}  
The semiring operations of addition and multiplication allow us to analyze combined procedures:
\begin{itemize}
    \item Addition $A + B$ combines subspaces (union of images) of two operators.
    \item Multiplication $AB$ corresponds to sequential application; non-commutativity $AB \neq BA$ quantifies incompatibility.
\end{itemize}
The sufficiency gap formalizes this incompatibility in terms of lost regression-relevant subspaces, making it a semiring-consistent measure of divergence between PCA and regression.
\subsection{Contradiction maps in operator space}

We further describe regions of operator-parameter space where regression- and 
PCA-type procedures coincide or diverge. Let $A \in P$ encode a Gaussian model
\[
Y = X \beta + \varepsilon, \qquad X \sim \mathcal{N}(0, \Sigma),
\]
with regression part $A_\beta$ and PCA projection part $A_\mu$.

Define the \emph{coincidence set} of operators
\[
\mathcal{C}_r := \{ A \in P : \operatorname{Im}(A_\beta) \subseteq \operatorname{Im}(A_\mu), \; \operatorname{rank}(A_\mu) = r \},
\]
i.e., the set of operators where the regression signal lies entirely within the PCA-projected subspace. In this region, regression and PCA-based reconstruction yield identical fitted values for the predictors.

The complement
\[
\mathcal{C}_r^c := P \setminus \mathcal{C}_r
\]
is the \emph{contradiction set}, where PCA projections discard part of the regression signal encoded in $A_\beta$.

\paragraph{Interpretation.}  
Visualizing $\mathcal{C}_r$ and $\mathcal{C}_r^c$ as subsets of the semiring $P$ produces \emph{contradiction maps}: geometric representations of where and why regression and PCA disagree in operator space.