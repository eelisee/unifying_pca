\section{Ideen}

\subsection{Chapter 6}

\subsubsection{Semiring Factorization of Inference Goals}

A key insight from the semiring perspective is that regression and PCA differ not in the underlying data model, but in their \emph{inferential goals}: regression targets conditional relationships, whereas PCA captures marginal structure.  
Our operator-semiring framework makes this distinction algebraically explicit: one can represent target quantities themselves as elements or ideals within the semiring.  

\begin{itemize}
    \item \textbf{Predictive vs. descriptive vs. protective goals:} The algebraic classification allows us to categorize inference objectives according to whether they aim at prediction, descriptive dimensionality reduction, or information protection.
    \item \textbf{Generalization to other methods:} Techniques such as Lasso, Ridge regression, or Canonical Correlation Analysis can be interpreted as specialized ideals or subsemirings within the same overarching operator framework, making apparent their structural relations to classical PCA and regression.
\end{itemize}

This factorization offers a unifying algebraic lens to compare and contrast diverse inferential targets within a single semiring structure.

\subsubsection{Robust Semiring Extensions}

Classical PCA is sensitive to heavy-tailed distributions (\(\alpha\)-stable with \(\alpha<2\)), whereas regression can remain well-defined.  
The semiring framework suggests a path to \emph{robust operator structures}:

\begin{itemize}
    \item Construct operator-semiring variants based on robust statistics, e.g., medians, median absolute deviations, or quantile-based projections.
    \item This yields an algebraic framework that unifies classical and robust procedures, preserving closure properties and semiring operations.
    \item Practical benefit: PCA-like dimension reduction and regression-based inference become robust to outliers and non-Gaussian data without leaving the semiring formalism.
\end{itemize}

\subsubsection{Universal Approximation via Operator Semirings}

A natural question is: which set of operators suffices to represent all meaningful inferential tasks, such as projections, regressions, or classifications, within this algebraic framework?  

\begin{itemize}
    \item \textbf{Universal representation:} Analogous to universal approximation theorems in analysis, one can envision a semiring-based universality result where PCA and regression operators serve as a \emph{basis set} from which complex inference procedures can be constructed.
    \item This perspective characterizes classical dimension reduction and predictive models as fundamental building blocks of a more general operator-algebraic inference system.
\end{itemize}

\subsubsection{Dynamic Semiring Evolution}

In real-world datasets, PCA subspaces and regression coefficients can change with sample growth or temporal evolution. The semiring formalism can be extended to capture such dynamics:

\begin{itemize}
    \item Define a \emph{time-dependent semiring} in which operators evolve continuously or discretely, forming a flow in a non-commutative algebraic space.
    \item Applications include online learning, streaming data analysis, and adaptive modeling, where both predictive and descriptive components of the model update coherently within the semiring.
\end{itemize}

This dynamic viewpoint integrates statistical adaptation into the algebraic structure, allowing the framework to reflect not only static inferential relationships but also their temporal evolution.


\subsection{Chapter 4: Sufficiency in regression- and PCA-type operator models}
\label{subsec:sufficiency}

The algebraic framework developed so far allows us to reinterpret sufficiency of statistics in terms of structured operators. Recall that for a Gaussian model, the empirical mean and covariance are sufficient for $(\mu,\Sigma)$. Within our semiring setting, sufficiency acquires a finer meaning: each operator $A \in P$ extracts exactly that part of the distribution of $\tilde{X}$ which is relevant for the parameters encoded in $A$.

\paragraph{Setup.}  
Let $\tilde{X} \sim \mathcal{N}(0, \Sigma)$ with $\Sigma \in \mathbb{R}^{k \times k}$ positive definite, and let $A \in P$. Define
\[
Y = T_A(\tilde{X}) := A \tilde{X}.
\]
Then
\[
Y \sim \mathcal{N}(0, A \Sigma A^\top).
\]
By the factorization theorem, $T_A(\tilde{X})$ is a sufficient statistic for any parameter of the Gaussian distribution that can be expressed as a functional of $A \Sigma A^\top$. In other words, $A$ defines a sufficient reduction of the data.

\paragraph{Regression-type operators.}  
For $A \in L_\ell$, the first row of $A$ encodes regression coefficients $\beta$ acting only on predictors $X_1,\dots,X_\ell$. The likelihood depends on $\beta$ solely through
\[
(X_1, \dots, X_\ell)^\top.
\]
Hence $T_A(\tilde{X})$ is sufficient for $\beta$ restricted to this subset. Algebraically, sufficiency coincides with the coordinate projection structure of $L_\ell$.

\paragraph{PCA-type operators.}  
For $A \in \mathcal{H}_r$, the block $A_\mu$ is a projection of rank $r$. The transformed variable $T_A(\tilde{X}) = A \tilde{X}$ depends on $\Sigma$ only through
\[
A_\mu \Sigma A_\mu^\top,
\]
which contains all information about the subspace preserved by $A_\mu$. Thus $T_A(\tilde{X})$ is sufficient for the projected covariance structure, i.e., for the eigenspaces corresponding to the chosen low-rank reconstruction, while discarding orthogonal directions.

\paragraph{Objective.}  
Show that \(T_A(\tilde{X})\) is a sufficient statistic for parameters of interest in a model class \(\mathcal{M}\) defined via a subset of the covariance structure of \(\tilde{X}\).

\paragraph{Likelihood-based Derivation.}  
The density of \(\tilde{X}\) is
\[
f_{\tilde{X}}(\tilde{x}; \Sigma) = \frac{1}{(2\pi)^{k/2} |\Sigma|^{1/2}} \exp\Big(-\frac{1}{2} \tilde{x}^\top \Sigma^{-1} \tilde{x}\Big), \quad \tilde{x} \in \mathbb{R}^k.
\]

Consider the transformation \(Y = A \tilde{X}\). Using the change-of-variables formula (valid for full-rank or rank-preserving \(A\)):
\[
f_Y(y; \Sigma) = \frac{1}{(2\pi)^{k/2} |A \Sigma A^\top|^{1/2}} \exp\Big(-\frac{1}{2} y^\top (A \Sigma A^\top)^{-1} y\Big), \quad y \in \operatorname{Im}(A).
\]

Notice that the density depends on \(y\) only through the quadratic form
\[
y^\top (A \Sigma A^\top)^{-1} y,
\]
i.e., on \(T_A(\tilde{X}) = A \tilde{X}\). By the factorization theorem, this implies $T_A(\tilde{X}) = A \tilde{X}$ is sufficient for all parameters encoded in $A$.

\paragraph{Comparison.}  
\begin{itemize}
    \item In regression ($L_\ell$), sufficiency is coordinate-aligned: only the selected predictors matter, and $T_A(\tilde{X})$ compresses the data to those coordinates without losing information about $\beta$.
    \item In PCA ($\mathcal{H}_r$), sufficiency is subspace-aligned: the sufficient statistic is the projection of $\tilde{X}$ onto a low-dimensional subspace, retaining all information about that subspace but losing coordinate-level interpretability.
\end{itemize}

\paragraph{Unified semiring perspective.}
Both regression and PCA thus admit sufficient statistics in the Gaussian model, but with different invariances: regression is invariant under scaling within selected coordinates, PCA under rotations within the chosen subspace. In the semiring language, this corresponds to sufficiency being stable under $L_\ell$ for regression and under $\mathcal{H}_r$ for PCA. The divergence between the two models therefore arises not from the lack of sufficiency, but from the fact that different operator families define different sufficient reductions of the same underlying distribution.